\section{Definitions}
\label{sec:definitions}


\subsection{Model}

We consider a system composed of $n$ server processes $\Pi = \{ 1, \dots, n\}$ that communicate by exchanging messages.
All messages sent between processes are digitally signed.
Processes can be \emph{correct} or \emph{faulty}, where correct processes
follow the algorithm and the faulty one may behave in an arbitrary way, i.e., we consider Byzantine faults.
We assume than the number of faulty servers is limited to $f$.

We consider a partially synchronous system model:
in all executions of the system, there is a bound $\Delta$
and an instant $GST$ (Global Stabilization Time) such that all
communication among correct processes after $GST$ is reliable and $\Delta$-timely, i.e., if a correct process
$p$ sends message $m$ at time $t\ge GST$ to correct process $q$, then $q$ will receive $m$
before $t+\Delta$.
We do not make any assumption
before $GST$. For example, messages among correct processes can be
delayed, dropped or duplicated before $GST$. Spoofing/impersonation attacks are
assumed to be impossible also before $GST$.

We assume that process steps (which might include
sending and receiving messages)
take zero time.
Processes are equipped with clocks able to measure local timeouts.

 \subsection{Consensus}
 \label{sec:consensus}

 \newcommand{\propose}{\mathsf{Propose}}
 \newcommand{\decide}{\mathsf{Decide}}

 In the consensus problem, every process has an initial value from a
      set $V$ and has eventually to irrevocably decide on a value from
      $V$.
 The problem is defined by an agreement, a termination, and a validity
      property.

 \begin{itemize}
 \item \emph{Agreement:} No two correct processes decide differently.
 \item \emph{Termination:} All correct processes eventually decide.
 \item \emph{Validity:} If all correct processes have the same initial value, and no process
 is faulty, then this is the only possible decision value.
 \end{itemize}
